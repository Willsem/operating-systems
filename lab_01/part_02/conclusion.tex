\section{Выводы}

И в ОС Windows, и UNIX обработчик системного таймера выполняет
схожие основные функции:

\begin{itemize}
    \item[--] обновление системного времени
    \item[--] уменьшение кванта процессорного времени,
        выделенного процессу
    \item[--] запуск планировщика задач
    \item[--] отправление отложенных вызовов на выполнение
\end{itemize}

Это обусловлено тем, что обе операционные системы являются
системами разделения времени с вытеснением и динамическими
приоритетами.

Однако в планировании семейства этих систем сильно различаются.
Классический Unix имеет невытесняющее ядро, а Windows является
полностью вытесняющей. Алгоритмы планирования имеют схожие черты
и основаны на очередях, но взаимодействия планировщика и потоков
в данных ОС имеют явные различия, к примеру, в Windows потоки сами
вызывают планировщик для пересчета их приоритетов.
